\documentclass[11pt,letterpaper,sans]{moderncv}   % possible options include font size ('10pt', '11pt' and '12pt'), paper size ('a4paper', 'letterpaper', 'a5paper', 'legalpaper', 'executivepaper' and 'landscape') and font family ('sans' and 'roman')

\moderncvstyle{banking}    % casual' (default), 'classic', 'oldstyle' and 
\moderncvcolor{blue}     % 'blue' (default), 'orange', 'green', 'red', 'purple', 'grey' and 'black'

\usepackage[utf8]{inputenc}
\definecolor{color1}{rgb}{0, 0, 0}


\usepackage[top=2cm,bottom=2cm,left=2cm,right=2cm,bindingoffset=0cm]{geometry}

\setlength{\hintscolumnwidth}{3cm}           % if you want to change the width of the column with the dates
%\setlength{\maketitlenamewidth}{10cm}     % for the 'classic' style, if you want to force the width allocated to your name and avoid line breaks.
%\usepackage{natbib}
%\def\@biblabel{\arabic{enumiv}} 




% customize the enumerate environments (i.e. enumerate, itemize, ...)
\usepackage{enumitem}
\setlist{nolistsep}



\firstname{Tao}
\familyname{Yang}
\title{Cover Letter}
%\email{dgerosa@caltech $\;\;\circ$ \today}
\phone{+86~15652195895}
\email{yangtao2017@bnu.edu.cn $\;\;\bullet\;\;$ \today}


\begin{document}
\recipient{Recruitment team}{}
\opening{Dear Sir or Madam,}
\closing{Yours faithfully,}
\makelettertitle

I am Tao Yang, here I am writing to inquire about the possibility of the postdoctoral position in your group. Currently, I am a postdoctoral fellow in Department of Astronomy, Beijing Normal University. My co-advisor is Professor Bin Hu.  I got my Ph.D degree in Institute of Theoretical Physics, Chinese
Academy of Sciences  with supervisor Rong-Gen Cai in 2017. My current research interests include mainly the gravitational waves, strong gravitational lensing, especially their applications on cosmology and test of gravity. I am looking forward to working with you in these related fields.

Having done some researches in cosmology and obtained rich experiences on data analysis methods such as
Gaussian process (GP) and Markov Chain Monte Carlo (MCMC), I turned to work on the researches about
using the gravitational waves as the standard sirens to probe the cosmology in the last two years of my PhD
period, and finished a series of works. Since 2018, I start to study and focus on the strong gravitational lensing physics. Recently, I cooperated with my
postdoctoral advisor and finished one paper on using the time-delay of strong lensed gravitational waves as a
new probe to test the gravity theory. The gravitational waves and the strong gravitational lensing related to
Cosmology and Astrophysics are the main topics which I will go on focusing in the future. I hope to learn more about the gravitational wave and strong lensing
knowledge related to theory and experiment from future’s cooperators and do some interesting works in these
fields.

\vspace{10mm}

\makeletterclosing



\end{document}
