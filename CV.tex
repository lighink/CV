\documentclass[letterpaper]{moderncv}


\usepackage{graphicx}
\usepackage{longtable}
\usepackage{multirow}
\usepackage{textcomp}
\usepackage{units}
\usepackage{lineno}
\usepackage{rotating}
\usepackage{amssymb}
\usepackage{amsmath}
\usepackage[utf8]{inputenc}
\usepackage{longtable}
\usepackage{comment}
\usepackage{xcolor}


\moderncvstyle{banking}
\moderncvcolor{blue}
\definecolor{color1}{rgb}{0.0, 0.45, 0.81}
\renewcommand{\familydefault}{\sfdefault}
\usepackage[top=2cm,bottom=2cm,left=2cm,right=2cm,bindingoffset=0cm]{geometry}
\setlength{\hintscolumnwidth}{3cm}
\usepackage{enumitem}
\setlist{nolistsep}

\firstname{Tao}
\familyname{Yang}
\title{Curriculum Vitae}
\phone{+86-15652195895}
\email{yangtao.lighink@gmail.com $\;\;\bullet$ \today}


%\extrainfo{  }
%{ }
%\homepage{davidegerosa.com}
%\photo[70pt][0.1pt]{myphoto}
% '64pt' is the height the picture must be resized to, 0.4pt is the thickness of the frame around it (put it to 0pt for no frame) and 'picture' is the name of the picture file; optional, remove the line if not wanted
%\quote{\small {I am relativistic astrophysicist, studying the impact of Einstein's general relativity on the astrophysical world.
%My research interests span from binary black-holes as gravitational-wave sources, to gas accretion onto black-holes, galaxy modelling and alternative theories of gravity.}}
% to show numerical labels in the bibliography (default is to show no labels); only useful if you make citations in your resume
\makeatletter
\renewcommand*{\bibliographyitemlabel}{\@biblabel{\arabic{enumiv}}}
\makeatother

\newcommand{\mnras}{Monthly Notices of the Royal Astronomical Society}
\newcommand{\mnrasl}{Monthly Notices of the Royal Astronomical Society Letters}
\newcommand{\prd}{Physical Review D}
\newcommand{\prl}{\textbf{Physical Review Letters}}
\newcommand{\prlplain}{{Physical Review Letters}}
\newcommand{\cqg}{Classical and Quantum Gravity}
\newcommand{\jcap}{Journal of Cosmology and Astroparticle Physics}
\newcommand{\aap}{Astronomy \& Astrophysics}


\long\def\suppress#1\endsuppress{%
  \begingroup%
    \tracinglostchars=0%
    \let\selectfont=\nullfont
    \nullfont #1\endgroup}

\fancypagestyle{headonly}{
\fancyfoot{}
\fancyfoot[r]{\textcolor{color2}{\thepage}}
\fancyhead{}
}

\begin{document}
\pagestyle{headonly}
\makecvtitle

\cvitem{}{\emph{\vspace{-1cm}\\
$\quad$ I am now a postdoctoral fellow, doing research on Cosmology and Gravitational Waves. My current interests include the theory and data analysis on Cosmology, Gravitational Waves Cosmology, and Gravitational Lensing, etc.}
}

\section{Personal information}
%\cvitem{Present position}{Ph.D. candidate.  \emph{University of Cambridge},
%\newline{}
%\protect{$~\qquad\qquad\qquad\quad\;\;\;$}
%Department of Applied Mathematics and Theoretical Physics (DAMTP),
%\newline{}
%\protect{$~\qquad\qquad\qquad\quad\;\;\;$}
%Centre for Mathematical Sciences, Wilberforce Road, Cambridge CB3 0WA, UK.}
\cvitem{Present position}{Postdoctoral fellow,
\newline{}
\protect{$~\qquad\qquad\qquad\quad\;\;\;$}
\emph{Department of Physics and Astronomy, Seoul National University},
\newline{}
\protect{$~\qquad\qquad\qquad\quad\;\;\;$}
1 Gwanak-ro, Gwanak-gu, Seoul 08826, Korea.}
%\cvitem{Personal webpage}{\href{http://www.tapir.caltech.edu/~dgerosa}{www.tapir.caltech.edu/\textasciitilde dgerosa}}
%\cvitem{Date of birth}{Sep 18, 1989.}
\cvitem{Citizenship}{Chinese.}
%\cvitem{Place of birth}{Sesto San Giovanni MI, Italy.}
%\cvitem{Personal address}{9 Lingholme Close, Cambridge CB4 3HW, UK.}
%\cvitem{Languages}{Chinese (native), English (fluent)}
%\cvitemwithcomment{English}{Fluent}{TOEFL exam:  109/120, Oct.2012}


\section{Research experience}

\cventry{2021.8-now}{Postdoctoral fellow}{Department of Physics and Astronomy, Seoul National University}{Seoul, Korea}{}{}
\vspace{-0.1cm}

\cventry{2019.8-2021.7}{Postdoctoral fellow}{Asia Pacific Center for Theoretical Physics}{Pohang, Korea}{}{}
\vspace{-0.1cm}

\cventry{2017.8-2019.7}{Postdoctoral fellow}{Department of Astronomy, Beijing Normal University}{Beijing, China}{}{}
\vspace{-0.1cm}
\begin{tabular}{rcl}
&\hspace{0.4cm} &$\circ\;\;${\textit{Additional support}}: Selected to be supported by  China Postdoctoral Science Foundation.
\end{tabular}

\vspace{0.2cm}
\cventry{2014.9-2017.6}{Ph.D. candidate}{Institute of Theoretical Physics, Chinese Academy of Sciences}{Beijing, China}{}{}
\vspace{-0.1cm}
\begin{tabular}{rcl}
&\hspace{0.4cm} &$\circ\;\;${\textit{Supervisor}}: Rong-Gen Cai.\\
&\hspace{0.4cm} &$\circ\;\;${\textit{Thesis}}: Gaussian Process and Gravitational Waves Cosmology
\end{tabular}

%\cventry{2014 - {current}}{Ph.D. candidate}{Institute of Theoretical Physics, Chinese Academy of Sciences}{Beijing, China}{}{}
%\vspace{-0.1cm}
%\begin{tabular}{rcl}
%&\hspace{0.4cm} &$\circ\;\;$%{\textit{Advisor}}: Ulrich Sperhake.\\
%&\hspace{0.4cm} &$\circ\;\;${\textit{Support}}: Isaac Newton Studentship and STFC Ph.D. Studentship.\\
%&\hspace{0.4cm} &$\circ\;\;${\textit{Thesis}}:  Source modelling at the dawn of gravitational-wave astronomy.\\
%&\hspace{0.4cm} &$\circ\;\;${\textit{Teaching assistant}}: Introduction to General Relativity (Part II, 3rd year undergraduate class);\\
%&\hspace{0.4cm} &\hspace{3.5cm} Advanced General Relativity (Part III, master class).\\
%&\hspace{0.4cm} &$\circ\;\;${\textit{Undergraduate summer project mentoring}}: J.~Vosmera, University of Cambridge, 2015;\\
%&\hspace{0.4cm} &\hspace{7.3cm} R.~Barbieri, University of Pavia, 2016.\\
%\end{tabular}


%\vspace{0.3cm}
%\cventry{Jun-Aug 2012}{LIGO Summer Undergraduate Research Fellow}{California Institute of Technology}{Pasadena CA, USA}{}{}
%%\cvitem{}{Scientific research within the LIGO SURF (Summer Undergraduate Research Fellowship) program. %Project developed as part of the LIGO Scientific Collaboration.}
%%\cvitem{}{Project title: Spin Alignment Effects in Black Hole Binaries.}
%%\cvitem{}{Advisor: Emanuele Berti.}
%\vspace{-0.1cm}
%\begin{tabular}{rcl}
%&\hspace{0.4cm} &$\circ\;\;${\textit{Advisor}}:  Emanuele Berti.\\
%\end{tabular}

%\vspace{0.3cm}
%\textbf{\textcolor{color1}{Extended research visits:}}\vspace{0.15cm}\\
%\begin{tabular}{rcl}
%&\hspace{0.4cm} &$\bullet\;\;${\textbf{Institut Astrophysique de Paris}}. Paris, France. \textit{Jul 2015}\\
%&\hspace{0.4cm} &$\bullet\;\;${\textbf{University of Mississippi}}. Oxford MS, USA. \textit{Aug-Dec 2012}\\
%&\hspace{0.4cm} &$\bullet\;\;${\textbf{Caltech}}. Pasadena CA, USA. LIGO Summer Undergraduate Research Fellow. \textit{Jun-Aug 2012}\\

%\end{tabular}


\section{Education}

\cventry{2012.9 - 2014.6}{Master's degree in Gravity theory and Cosmology}{\newline Institute of Theoretical Physics, Chinese Academy of Sciences}{Beijing, China}{}{}
%\vspace{-0.1cm}
%\begin{tabular}{rcl}
%&\hspace{0.4cm} &$\circ\;\;${\textit{Final degree grade}}: 110/110 with distinction (``cum laude''),\\
%&\hspace{0.4cm} &$\circ\;\;${\textit{Average class grade}}: 30/30 with 7/12 distinctions (``cum laude''). Top 1\% of my class.\\
%&\hspace{0.4cm} &$\circ\;\;${\textit{Thesis advisors}}: Giuseppe Lodato, Emanuele Berti.\\
%\cvitem{}{Master's thesis: \emph{Black hole spin alignment in astrophysical environments}.}
%\end{tabular}


\vspace{0.2cm}
\cventry{2008.9 - 2012.6}{Bachelor's degree in Physics}{Wuhan University}{Wuhan, China}{}{}
\vspace{-0.1cm}
\begin{tabular}{rcl}
&\hspace{0.4cm} &$\circ\;\;${\textit{Department}}: Physics base classes,  School of Physics and Technology.\\
&\hspace{0.4cm} &$\circ\;\;${\textit{Graduate}}: Postgraduate Candidates Exempt from Admission Exam (Top 30\%).\\
&\hspace{0.4cm} &$\circ\;\;${\textit{Thesis advisor}}:  Jue-Ping Liu.\\
%\cvitem{}{Bachelor's thesis: \emph{Physical applications of Finsler geometry}.}
\end{tabular}

%\vspace{0.3cm}
%\cventry{2002 - 2007}{Scientific High School degree}{\newline Liceo Scientifico Statale Paolo Frisi}{Monza MB, Italy}{}{}  %
%\vspace{-0.1cm}
%\begin{tabular}{rcl}
%&\hspace{0.4cm} &$\circ\;\;${\textit{Final grade}}:  100/100.\\
%\end{tabular}

\section{Grants, scholarships and awards}

%\cventry{2016 - {current}}{NASA \& Smithsonian Astrophysical Observatory (Harvard)}{Einstein Postdoctoral Prize Fellowship}{Pasadena CA, USA}{}{}
\cvitemwithcomment{}{\textbf{Young Scientist Training Program (YST)}, APCTP}{2019}
\cvitemwithcomment{}{\textbf{Top 100 doctoral dissertations of Chinese Academy of Sciences}, UCAS}{2018}
\cvitemwithcomment{}{\textbf{Postdoctoral Science Fund support}, China Postdoctoral Science Foundation}{2017}
\cvitemwithcomment{}{\textbf{Outstanding graduates of UCAS}, University of Chinese Academy of Sciences}{2017}
\cvitemwithcomment{}{\textbf{National scholarship for Doctoral students (Top 3\%)}, University of Chinese Academy of Sciences}{2016}
\cvitemwithcomment{}{\textbf{Institute Studentship}, University of Chinese Academy of Sciences}{2016-2017}
\cvitemwithcomment{}{\textbf{Institute Studentship}, University of Chinese Academy of Sciences}{2015-2016}
\cvitemwithcomment{}{\textbf{Undergraduate Studentship},Wuhan University}{2010-2011}


%\vspace{0.05cm}
%\cventry{2015}{Faculty of Mathematics}{Smith-Rayleigh-Knight Essay Prize}{Cambridge, UK}{University of Cambridge}{}

%\vspace{0.05cm}
%\cventry{2014-2015}{Darwin College}{Travel fund}{Cambridge, UK}{University of Cambridge}{}
%\vspace{0.05cm}
%\cventry{2015}{Cambridge Philosophical Society}{Travel fund}{Cambridge, UK}{University of Cambridge}{}

%\vspace{0.05cm}
%\cventry{2012}{LIGO Collaboration}{LIGO Summer Undergraduate Research Fellowship}{Pasadena CA, USA}{California Institute of Technology}{}

%\vspace{0.05cm}
%\cventry{2007 - 2008}{Physics Department, University of Milan}{Undergraduate Studentship}{Milan, Italy}{}{}

%\cvitem{Nov 2011}{Award assigned by the cultural association "Famiglia Legnanese" to  best Italian College students, Legnano MI, Italy.}
%\cvitem{Jul 2007}{Award assigned by the website \textit{matematicamente.it} to the best italian high-school research project, presenting my work  {"Do I Dare disturb the Universe?"} about  evidences for dark matter.}
%\newpage
\section{Publications}
%\vspace{0.2cm}

%\cvitem{}{
%\begin{tabular}{rcl}
%\textbf{Publication counts}: &\hspace{0.3cm} &{\textbf{14} papers published in peer-reviewed journals} \\
%& &{(out of which \textbf{9} first-authored)}
%\\
%& &{\textbf{1} submitted paper}
% \\
%& &{\textbf{1} paper in conference proceedings}
%\end{tabular}
%}
%\textbf{Total number of citations:} 328 (using ADS and InSpire)

%\textbf{h-index:} 10
%\textbf{Web links to list services:}
%\href{http://labs.adsabs.harvard.edu/adsabs/search/?q=author%3A%22Gerosa%2C+Davide%22&month_from=&year_from=&month_to=&year_to=&db_f=&nr=&article=1&bigquery=&re_sort_type=CITED&re_sort_dir=desc}{\textsc{ADS}};
%\href{http://inspirehep.net/search?ln=en&ln=en&p=exactauthor%3AD.Gerosa.1&of=hb&action_search=Search&sf=&so=d&rm=citation&rg=25&sc=0}{\textsc{InSpire}};
%\href{http://arxiv.org/a/gerosa_d_1.html}{\textsc{arXiv}}.

%\textbf{Full list of publications available at:} \href{http://www.tapir.caltech.edu/~dgerosa/pub}{www.tapir.caltech.edu/\textasciitilde dgerosa/pub}



%%%%%%%%%%%%%%%%%%%%%%%%%%%%%%%%%
%%%%%%%%%%%%%%%%%%%%%%%%%%%%%%%%%
%%%%%%%%%%%%%%%%%%%%%%%%%%%%%%%%%

Total:23

Citations:711,
h-index: 14

\textbf{$\bullet$} As the first contributor

\vspace{+0.2cm}
\cvitem{List of publications}{}
\vspace{-0.7cm}

\cvitem{}{\small
\hspace{-1cm}\begin{longtable}{rp{0.3cm}p{15.55cm}}
\textbf{$\bullet$} & & Space-borne Atom Interferometric Gravitational Wave Detections II: Dark Sirens and Finding the One,
\textbf{T.~Yang}, H. M. Lee, R. G. Cai, H. Choi, S. Jung,
arXiv:2110.09967.
\vspace{0.05cm}\\
\textbf{$\bullet$} & & Space-borne Atom Interferometric Gravitational Wave Detections: The Forecast of Bright Sirens on Cosmology,
R. G. Cai, \textbf{T.~Yang},
arXiv:2107.13919.
\vspace{0.05cm}\\
\textbf{$\circ$} & & The Gravitational-Wave Physics II: Progress,
L. Bian, R. G. Cai, S. Cao, Z. Cao, di Li, H. Gao, Z. Guo, K. Lee, J. Liu, Y. Lu, S. Pi, J. Wang, S. Wang, Y. Wang, \textbf{T. Yang}, X. Yang, S. Yu, X. Zhang,
Sci.China Phys.Mech.Astron. 64 (2021) 120401,
arXiv:2106.10235.
\vspace{0.05cm}\\
\textbf{$\bullet$} & & Gravitational-Wave Detector Networks: Standard Sirens on Cosmology and Modified Gravity Theory,
\textbf{T.~Yang},
JCAP 05 (2021) 044,
arXiv:2103.01923.
\vspace{0.05cm}\\
\textbf{$\circ$} & & Running Hubble Tension and a H0 Diagnostic,
C. Krishnan, E. Ó. Colgáin, M. M. Sheikh-Jabbari, \textbf{T.~Yang},
 Phys.Rev.D 103 (2021) 103509,
 arXiv:2011.02858.
\vspace{0.05cm}\\
\textbf{$\circ$} & & On cosmography in the cosmic dark ages: are we still in the dark?
A. Banerjee, E. Ó. Colgáin, M. Sasaki, M. M. Sheikh-Jabbari, \textbf{T.~Yang},
 Phys.Lett.B 818 (2021) 136366,
 arXiv:2009.04109.
\vspace{0.05cm}\\
\textbf{$\bullet$} & & Model-Independent Perspectives on Coupled Dark Energy and the Swampland,
 \textbf{T.~Yang},
 Phys.Rev.D 102 (2020) 083511,
 arXiv:2006.14511.
\vspace{0.05cm}\\
\textbf{$\circ$} & & Hubble Sinks In The Low-Redshift Swampland,
  A.~Banerjee, H.~Cai, L.~Heisenberg, E.~Ó.~Colgáin, M.~Sheikh-Jabbari and \textbf{T.~Yang},
  Phys.Rev.D 103 (2021) L081305,
  arXiv: 2006.00244.
\vspace{0.05cm}\\
\textbf{$\bullet$} & & The first simultaneous measurement of Hubble constant and post-Newtonian parameter from Time-Delay Strong Lensing,
  \textbf{T.~Yang}, S.~Birrer and B.~Hu, 
  Mon.Not.Roy.Astron.Soc. 497 (2020) 1, L56-L61,
  arXiv: 2003.03277.
\vspace{0.05cm}\\
\textbf{$\circ$} & & Is there an early Universe solution to the Hubble tension?
  C.~Krishnan, E.~Ó.~Colgáin, Ruchika, A.~A.~Sen, M.~M.~Sheikh-Jabbari and \textbf{T.~Yang},
  Phys.Rev.D 102 (2020) 103525,
  arXiv: 2002.06044.
\vspace{0.05cm}\\
\textbf{$\bullet$} & & Gaussian processes reconstruction of modified gravitational wave propagation,
  E.~Belgacem, S.~Foffa, M. Maggiore, and \textbf{T.~Yang},
  Phys.Rev. D101 (2020) 063505,
  arXiv: 1911.11497.
\vspace{0.05cm}\\
\textbf{$\bullet$} & & On cosmography and flat $\Lambda$CDM tensions at high redshift,
  \textbf{T.~Yang}, A. Banerjee, E. O. Colgain,
  Phys.Rev.D 102 (2020) 123532,
  arXiv: 1911.01681.
\vspace{0.05cm}\\
\textbf{$\bullet$} & & New probe of gravity: strongly lensed gravitational wave multi-messenger approach,
  \textbf{T.~Yang}, B.~Hu, R.~G.~Cai and B.~Wang,
  Astrophys.J. 880 (2019) 50,
  arXiv:1810.00164.
\vspace{0.05cm}\\
\textbf{$\bullet$} & & Super-Eddington accreting massive black holes explore high-$z$ cosmology: Monte-Carlo simulations,
  R.~G.~Cai, Z.~K.~Guo, Q.~G.~Huang and \textbf{T.~Yang},
  Phys.Rev. D97 (2018) 123502, 
  arXiv:1801.00604.
\vspace{0.05cm}\\
\textbf{$\circ$} & & Probing cosmic anisotropy with gravitational wave as standard siren,
  R.~G.~Cai, T.~B.~Liu, X.~W.~Liu, S.~J.~Wang and \textbf{T.~Yang},  
  Phys.Rev. D97 (2018) 103005,   
  arXiv:1712.00952. 
\vspace{0.05cm}\\
\textbf{$\bullet$} & & Constraints on the cosmic distance duality relation with simulated data of gravitational waves from the Einstein Telescope,
  \textbf{T.~Yang}, R.~F.~L.~Holanda and B.~Hu,
  Astropart.Phys. 108 (2019) 57-62,
  arXiv:1710.10929.
\vspace{0.05cm}\\
\textbf{$\bullet$} & & Standard sirens and dark sector with Gaussian process,
  R.~G.~Cai and \textbf{T.~Yang},
  EPJ Web Conf. 168 (2018) 01008,
  arXiv:1709.00837.
\vspace{0.05cm}\\
\textbf{$\bullet$} & & Reconstructing the dark sector interaction with LISA,
  R.~G.~Cai, N.~Tamanini and \textbf{T.~Yang},
  JCAP 1705 (2017) 031.
  arXiv:1703.07323.
\vspace{0.05cm}\\
\textbf{$\circ$} & &  The Gravitational-Wave Physics,
  R.~G.~Cai, Z.~Cao, Z.~K.~Guo, S.~J.~Wang and \textbf{T.~Yang},
  Natl.Sci.Rev. 4 (2017) 687-706,
  arXiv:1703.00187.
\vspace{0.05cm}\\
\textbf{$\bullet$} & & Estimating cosmological parameters by the simulated data of gravitational waves from the Einstein Telescope,
  R.~G.~Cai and \textbf{T.~Yang},
  Phys.Rev. D95 (2017) 044024,
  arXiv:1608.08008.
\vspace{0.05cm}\\
\textbf{$\bullet$} & &  Dodging the cosmic curvature to probe the constancy of the speed of light,
  R.~G.~Cai, Z.~K.~Guo and \textbf{T.~Yang},
  JCAP 1608 (2016) 016,
  arXiv:1601.05497.
\vspace{0.05cm}\\
\textbf{$\bullet$} & & Null test of the cosmic curvature using $H(z)$ and supernovae data,
  R.~G.~Cai, Z.~K.~Guo and \textbf{T.~Yang},
  Phys.Rev. D93 (2016) 043517,
  arXiv:1509.06283.
\vspace{0.05cm}\\
\textbf{$\bullet$} & & Reconstructing the interaction between dark energy and dark matter using Gaussian Processes,
  \textbf{T.~Yang}, Z.~K.~Guo and R.~G.~Cai,
  Phys.Rev. D91 (2015) 123533,
  arXiv:1505.04443.
\vspace{0.05cm}\\
\end{longtable}
}


%\section{Presentations}
%\vspace{0.1cm}

%\cvitem{}{
%\begin{tabular}{rcl}
%\textbf{Presentation counts}: &\hspace{0.3cm} &{\textbf{11} talks at conferences} \\
%& &{\textbf{11} talks at department seminars}
% \\
%& &{\textbf{7} posters at conferences}
%\end{tabular}
%}

%\textbf{Full list of presentations available at:} \href{http://www.damtp.cam.ac.uk/user/dg438/talks/index.html}{www.damtp.cam.ac.uk/user/dg438/talks}



%\vspace{+0.2cm}
%\cvitem{Selected presentations}{}
%\vspace{-0.7cm}

%\cvitem{}{\small
%\hspace{-1cm}\begin{longtable}{rp{0.3cm}p{15.55cm}}
%\textbf{$\bullet$} & & \textit{Averaging the average: multi-timescale analysis of precessing black-hole binaries}
%\newline{}21st International Conference on General Relativity and Gravitation (GR21), New York NY, USA, Jul 2016.
%\vspace{0.05cm}\\
%\textbf{$\bullet$} & & \textit{A new instability to black-hole spin precession}
%\newline{}28th Texas Symposium on Relativistic Astrophysics, Geneva, Switzerland, Dec 2015.
%\vspace{0.05cm}\\
%\textbf{$\bullet$} & & \textit{Analytic solutions to binary black-hole spin precession: recalling Kepler's two-body problem.}
%\newline{}Compact Objects as Astrophysical and Gravitational Probes, Leiden, The Netherlands, Jan 2015.  \newline{} \textbf{Best young presentation award}
%\vspace{0.05cm}\\
%\textbf{$\bullet$} & & \textit{Rival families: waveforms from resonant BH binaries as probes of their astrophysical formation history.}
%\newline{}3rd Session of the Sant Cugat Forum on Astrophysics, San Cugat, Spain, Apr 2014.  \end{longtable}
%}


\section{Computer Skills}
%\cvitem{}{$\bullet$ Developer of open-source Python module \textsc{precession}: \href{https://github.com/dgerosa/precession}{github.com/dgerosa/precession}. Complete toolbox to study the post-Newtonian dynamics of  spinning black-hole binaries (see Gerosa \& Kesden 2015).}
\cvitem{Operating systems}{Linux, Windows, MacOS}
\cvitem{Coding}{Python, Mathematica,  Fortran, Bash shell%C, Fortran, (also with Root, RooFit and Numerical Recipes tools),\newline
%Bash
.}
\cvitem{Data analysis experience}{Gaussian Process, Markov Chain Monte Carlo (CosmoMC, Cobaya, emcee), Gravitational waves simulation (and forecasting), Strong gravitational lensing etc.}
\cvitem{Data usage}{Planck CMB, BAO, Supernovae, Cosmic Chronometers, Gravitational Waves, Quasars,  Strong gravitational lensing (H0LiCOW) etc.}

%\section{Outreach and Service}

%\cvitemwithcomment{}{Student Member of the Gravitational Physics committee of the UK Institute of Physics (IoP).}{2014-2016}
%\cvitemwithcomment{}{Organizing committee of the international conference "Einstein's Legacy", London Queen Mary.}{2015}
%\cvitemwithcomment{}{Actively involved in the Cambridge Science Festival, faculty of Mathematics.}{2015}
%\cvitemwithcomment{}{Research coverage by the Caltech outreach journal: CURJ, Vol.15 No.1 (2014).}{2014}
%\cvitemwithcomment{}{Scientific journalist for the Italian on-line newspaper \href{http://www.ilsussidiario.net}{\textit{ilsussidiario.net}}}{2013-now}

%\cvitemwithcomment{}{Author in an undergraduate lecture-notes textbook on "Waves and Oscillations" (edited by CUSL press)}{2008}
%\mbox{(\url{euresis.org})}.
%I contributed to the preparation of scientific outreach exhibitions presented at the \textit{Rimini Meeting}, a one-week cultural event visited by more than 800.000 people every year.}
%\vspace{-1\baselineskip}
%\cvitem{}{\small
%\begin{itemize}
%\item Is the atom really indivisible? Questions and certainties in science.
%\item From one to infinity. At the heart of mathematics.
%\item Things never seen before. Galileo, the struggle and wonder of a new gaze on the Universe.
%\item Atmosphera. Reality and myth of global changes
%\end{itemize}}
%\vspace{-1\baselineskip}
%
%
%\section{Extracurricular Activities}
%\cventry{2014-now}{PhD student representative}{Gravitational Physics Group, UK Institute of Physics}{London, UK}{}{}
%
%\cventry{2008-2012}{Elected undergraduate student representative}{Science Faculty Council and Physics Coordination Committee}{Milan, Italy}{University of Milan}{}
%
%\cventry{2004-2006}{High-school student representative, elected in public elections}{Regional High School Student Assembly}{Milan, Italy}{Milan city council, Italy}{}
%
%
%\section{Work Experiences}
%\cventry{2013}{Liceo Don Gnocchi}{High-school maths teacher}{Carate Brianza MB, Italy}{}{}
%\cventry{2013}{CEUR foundation}{Maths and physics tutoring for undergraduate students.}{Milan, Italy.}{}{}
%
%\cventry{2007-2012}{CUSL press}{Physical sciences publications manager.}{Milan, Italy.}{}{}{}
%
%\cvitemwithcomment{}{Introductory \emph{Astronomy} lectures to high-school classes}{{2011-2012}}%Liceo Linguistico M.Candia, Seregno MB, Italy; Collegio della Guastalla, Monza MB, Italy.}
%%\cvitem{2010}{IT Laboratory keeper; Department of Physics, Universit\`{a} degli Studi di Milano, Milan, Italy.}
%%\cvitem{2007 - 2013}{Long lasting experience in private tutoring sessions for high-school students (maths and physics).}
%\cvitemwithcomment{}{Private tutoring sessions for high-school students (maths and physics).}{2007-2013}
%
%%\cvitem{2007 - 2012}{Shop assistant and book orders, responsible for the publications in the Physics branch; CUSL press and bookstore, Milan, Italy.}
%
%\section{Hobbies}
%\cvitem{}{Alpinism, rock climbing, alpine skiing and ski touring.}
%\cvitem{}{Soccer, Table tennis, etc.}
%\cvitem{}{Rock music}%{}{}{I reached the top of Monte Rosa (4554 meters, Italy) and Piz Palù (3882 meters, Switzerland)}{}

%
%\section{References}
%\cvitem{}{\textbf{Ulrich Sperhake}\newline{}Department of Applied Mathematics and Theoretical Physics, \newline{}University of Cambridge, \newline{}Wilberforce Road, Cambridge CB3 0WA, UK}
%\cvitem{}{\textbf{Emanuele Berti}\newline{}Department of Physics and Astronomy, \newline{}The University of Mississippi, \newline{}University, MS 38677, USA}
%\cvitem{}{\textbf{Giuseppe Lodato}\newline{}Dipartimento di Fisica, \newline{}Universit\`{a} degli studi di Milano, \newline{}Via Celoria 16, Milano, 20133, Italy}
%
\end{document}